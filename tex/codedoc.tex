\documentclass{article}

\usepackage{amsmath,amssymb,amsthm}

% Layout
\usepackage{geometry}
\geometry{
  left=2.5cm,
  right=2.5cm,
  top=2cm,
  bottom=4cm,
  bindingoffset=5mm
}

\usepackage[numbered,framed]{matlab-prettifier}

% Headers
\usepackage{fancyhdr}
\pagestyle{fancy}
\fancyhf{}

\chead[fMRI joint reconstruction]{fMRI joint reconstruction}

\newcommand{\mat}[1]{\lstinline[style=Matlab-editor]{#1}}

\newcommand{\TV}{\mathrm{TV}}
\newcommand{\C}{\mathbb{C}}
\newcommand{\R}{\mathbb{R}}
\newcommand{\Rplus}{\mathbb{R}_+}
\newcommand{\ICBTV}{\mathrm{ICB}_\TV}

\newcommand{\x}{\mathbf{x}}
\newcommand{\ubold}{\mathbf{u}}
\newcommand{\vbold}{\mathbf{v}}

\newcommand{\pa}{p^{\mathrm{art}}}
\newcommand{\qa}{q^{\mathrm{art}}}

\newcommand{\diverg}{\mathrm{div}}

\newcommand{\Kcal}{\mathcal{K}}
\newcommand{\Lcal}{\mathcal{L}}
\newcommand{\Fcal}{\mathcal{F}}
\newcommand{\Scal}{\mathcal{S}}
\newcommand{\Pcal}{\mathcal{P}}

\newcommand{\Real}{\mathrm{Re}}
\newcommand{\Imag}{\mathrm{Im}}

\newcommand{\qt}{\tilde{q}}


% Environments
\newtheorem{mydef}{Definition}
\newtheorem{mythm}{Theorem}
\newtheorem{myprop}{Proposition}
\newtheorem{mylem}{Lemma}
\newtheorem{myrem}{Remark}

% lstlistings
\lstset{
  style              = Matlab-editor,
%   basicstyle         = \mlttfamily,
  escapechar         = ",
  mlshowsectionrules = true,
}

\begin{document}
\title{Code documentation \\ Version 1.4} 
\author{Julian Rasch}

\maketitle

% \section{General setting}
% Let us first fix the notation.
% \begin{enumerate}
%  \item The inverse problem for the anatomical prior $u_0$ is $\Kcal_0 u_0 = f_0$.
%       Its necessary subgradient is $p_0 \in \partial \TV(u_0)$.
%    The sampling operator $\Kcal_0$ should provide a sampling such that the $u_0$ is a really good reconstruction. 
%  \item The inverse problems for the series are $\Kcal_i u_i = f_i$ for $i = 1, \dots, T$.
%    The sampling operators here should do undersampling of the data $f_i$.
% \end{enumerate}

\section{The anatomical prior and its reconstruction}
We shall start with an explanation of the acquisition and reconstruction scheme for the anatomical prior.

\subsection{Operator modeling and the method}
In order to clarify the code, we need to further specify the sampling operator for the anatomical prior.
We follow the ideas of [Ehrhardt, Betcke].
First, and most important, we restrict the prior to be real-valued (otherwise it is not directly clear how to define gradients, subgradients etc. for the reconstruction, see remark below), i.e. $u_0 \in \R^{N_0}$ . 
However, since the Fourier transform acts on complex-valued images, we need to embed our image into the complex numbers $\C^{N_0}$ via the embedding operator 
\begin{align*}
 i \colon \R^{N_0} \to \C^{N_0}, u_0, \quad u\mapsto u + 0i.
\end{align*}
It is the adjoint operator of the real part restriction, i.e. $i^*(x + iy) = \Real(x + iy) = x$, for $x,y \in \R^{N_0}$. 
We then apply a standard Cartesian Fourier transform to the embedded image. 
In order to simulate an arbitrary MR acquisition protocol, we do not use all the Fourier coefficients, but a sampling operator $\Scal_0 \colon \C^{N_0} \to \C^{M_0}$ to only choose a subset of the Fourier coefficients. 
For practical use, only the situation of undersampling, i.e. ${M_0} \ll {N_0}$ is relevant. 
Hence, if we let $\Pcal \colon \{1,\dots,{M_0}\} \to \{1,\dots,{N_0}\}$ be a mapping which chooses ${M_0}$ Fourier coefficients from the $N_0$ coefficients available, we can define the sampling operator $\Scal_0$ applied to $z \in \C^{N_0}$ 
\begin{align*}
 \Scal_0 \colon \C^{N_0} \to \C^{M_0}, \quad (\Scal_0 x)_k = z_{\Pcal(k)}.
\end{align*}
The full forward operator $\Kcal_0$ can hence be expressed as 
\begin{align}
\label{eq:pr_forward_op}
 \Kcal_0 \colon \R^{N_0} \xrightarrow{i} \C^{N_0} \xrightarrow{\Fcal} \C^{N_0} \xrightarrow{\Scal_0} \C^{M_0}.
\end{align}
Its adjoint is 
\begin{align}
\label{eq:pr_adjoint_op}
 \Kcal_0^* \colon \C^{M_0} \xrightarrow{\Scal_0^*} \C^{N_0} \xrightarrow{\Fcal^{-1}} \C^{N_0} \xrightarrow{i^*} \R^{N_0},
\end{align}
where $\Scal_0^*$ `fills' the missing frequencies with zeros, i.e. 
\begin{align*}
 \Scal_0^*(z) = \sum_{k=1}^{M_0} z_k \delta_{k,\Pcal(k)}.
\end{align*}
%
\begin{myrem}
 Note that applying $\Scal_0^*$ is equivalent to the minimum norm solution restricted to the real part, hence can be used as a `standard inversion' for comparison.
\end{myrem}


In order to get a piecewise constant reconstruction of the prior to use its gradient information, we perform a total variation regularized reconstruction. 
\begin{align}
   \min_{u_0 \in \R^{N_0}} \frac{\alpha_0}{2} \| \Kcal_0 u_0 - f_0\|_\C^2 + \TV(u_0),
   \label{eq:l2_tv}
\end{align}
where $\TV(u_0) = \| \nabla u_0 \|_1$ with a discrete gradient operator $\nabla \colon \R^{N_0} \to \R^{N_0 \times 2}$ (cf. [Chambolle]).
Hence 
\begin{align*}
 \TV(u_0) = \| \nabla u_0 \|_1 = \sum_{k=1}^{N_0} \sqrt{ |\nabla u_0|^2_{k,1} + |\nabla u_0|^2_{k,2} }.
\end{align*}
After the reconstruction, we can update the subgradient via the optimality condition: 
\begin{align}
 p_0 = \frac{1}{\alpha_0} \Kcal_0^* ( f_0 - \Kcal_0 u_0).
 \label{eq:pr_subgradient}
\end{align}
%
\begin{myrem}
 It is also possible to define the operator $\Kcal_0$ not on $\R^{N_0}$ but on $\C^{N_0}$, and perform the reconstruction with the constraint that $u_0$ is real-valued. 
 However, then there is no accessible optimality condition which allows to get a subgradient. 
\end{myrem}
%
A general problem with this subgradient is, due to its relation to the residual, it contains noise. 
For standard Bregman iterations this is desirable [Burger, Osher], however, here it is questionable, why `noise' should be transferred from the prior to the following dynamic reconstruction.
Therefore, we provide an alternative solution, an artificial `subgradient'.
Recall, that $p_0 \in \partial \TV(u_0)$ if and only if there exists $q_0 \in \partial \|\nabla u_0\|_1$ such that $p_0 = - \diverg(q_0)$.
In particular we have that 
\begin{align*}
 q_0 \begin{cases}
        = \nabla u_0 / |\nabla u_0|, &\text{ if } \nabla u_0 \neq 0, \\
        \in B(0,1), &\text{ else. }
       \end{cases}
\end{align*}
Hence, having a good reconstruction $u_0$ of the prior at hand, we can create an artificial one by letting 
\begin{align*}
 \qt_0 = \begin{cases}
        \nabla u_0 / |\nabla u_0|, &\text{ if } \nabla u_0 \neq 0, \\
        0, &\text{ else. }
       \end{cases}
\end{align*}
However, in view of not entirely perfect reconstructions in the numerical situation, it is preferable to decide, which height a jump has to have to become an edge.
Hence, introducing the edge parameter $\eta$, we have 
\begin{align*}
 \qt_{0,\eta} = \begin{cases}
        \nabla u_0 / |\nabla u_0|, &\text{ if } |\nabla u_0| \geq \eta, \\
        0, &\text{ else. }
       \end{cases}
\end{align*}
Then, finally, $p_0 = -\diverg(q_0)$ or $\tilde{p}_{0,\eta} = -\diverg (\qt_{0,\eta})$, respectively. 
We shall however simply refer to the subgradient as $p_0$, independently of the choice of $q_0$.
Comparing the different choices of gradients we notice, that the `artificial' subgradients are basically a `clean' version of $p_0$.

\subsection{Numerical realization}
The above considerations can be found in this code toolbox. 
We shall start with the operator $\Kcal_0$ and its adjoint $\Kcal_0^*$.

\begin{itemize}
 \item \mat{sampling_geom.m:} This function can create a lot of possible sampling pattern for the MR as a binary map, i.e. represents the operator $\Scal_0$. 
 For the prior, good choices are a \mat{'full'} sampling, or a sufficiently dense sampling on radial \mat{'spokes'} through the $k$-space center. 
 The sampling can be created with the call \mat{S\_0 = sampling\_geom(u0,'full');} or \mat{S\_0 = sampling\_geom(u0,'spokes1','nSpokes',n);}, where \mat{n} is the desired number of spokes. 
 Other samplings can be called identically, just check the documentation of the function, including a lot of options like number of sampling points and many more. 
 The input \mat{u0} is only used for the size of the Fourier transform, hence can also be a dummy.
 
 \item \mat{sampling_op.m:} Once you have generated a sampling for $\Scal_0$ via \mat{sampling_geom.m}, you can create a function handle for the operator $\Kcal_0$, i.e.  a Fourier transform followed by a projection onto the frequencies specified in $\Scal_0$ (or \mat{S_0}). 
 The syntax is \mat{K\_0 = @(x) sampling_op(x,S_0);}.
 
 \item \mat{sampling_adj.m:} This function is the adjoint operator $\Kcal_i^*$, and can be called analogously.
 
 \item \mat{sampling_vis.m:} This function can visualize the sampling via \mat{sampling_vis(S_0);}.
\end{itemize}

If we use artificial data made from a phantom, we can use the forward operator $\Kcal_0$ to create some data $f_0$ from the clean phantom $u_0^{\mathrm{clean}}$. 
\begin{itemize}
 \item \mat{fourier_data.m:} This function creates noisy and undersampled Fourier data from a clean phantom $u_0^{\mathrm{clean}}$.   
 The ratio of energy of the noise and energy of the data can be chosen by the parameter \mat{energy}. 
 The function call is \mat{f0 = fourier_data(K_0\{1,1\},u0_clean,energy);}.
 You can also visualize the data and how it is sampled via \mat{sampling_vis(f0);}.
\end{itemize}
Finally we can do the reconstruction \eqref{eq:l2_tv}. 
\begin{itemize}
 \item \mat{l2_tv.m:} This function solves \eqref{eq:l2_tv} with a primal-dual scheme and monitors the decrease of the primal energy as well as the primal-dual residual. 
 We can set a few options via the script \mat{paramfile_tv.m}. 
 \lstinputlisting{matlab_includes/c_paramfile_tv.m}
 The stopping criterion is a sufficiently small primal-dual residual [Ref], we can increase the accuracy of the solution by lowering the tolerance \mat{pdtol}.
 The value of \mat{niter} controls the amount of iterations, while \mat{int} controls the intervals in which the primal energy and the primal-dual residual is evaluated.
 If \mat{show} is set to \mat{true}, the function plots the progress of the energy and the residual after convergence.
 The parameters \mat{tau} and \mat{sigma} are the step sizes for the algorithm.
\end{itemize}

\begin{myrem}
 We would like to comment on the choice of the step sizes $\tau$ and $\sigma$. 
 In theory, they need to be chosen such that the condition $\tau \sigma L^2 < 1$ is fulfilled, where $L$ denotes the operator norm of the linear operator in the formulation of the used primal-dual scheme (cf. [Chambolle-Pock]).
 However, in our case, especially when using function handles and operators, that are not available in matrix form, it is not entirely clear how to determine $L$. 
 This is in particular difficult for the algorithm in the next section, which involves a combination of block operators (cf. [Rasch] for a version with two blocks). 
 Hence, we determine \mat{tau} and \mat{sigma} empirically. 
 We just choose a value and control the convergence, if it oscillates, we decrease both parameters by a factor and try again.
 However, I am working on a method to determine it automatically.
\end{myrem}


After convergence, we need to compute a subgradient, either via the optimality condition, or an artificial one. 
This is done via the function \mat{artificial_subgradient.m}.
The whole procedure is summarized in the following code: 
\lstinputlisting{matlab_includes/c_prior.m}
%
It is interesting to visually compare \mat{p_true} and \mat{p_art}, in particular it gives an indicator of how big we need to choose the edge parameter \mat{eta}. 
Just make sure that \mat{p_art} resembles \mat{p_true} without noise.


\section{The dynamic scan}  
Now we can proceed to the dynamic scans, which basically follow the same idea as before. 
The operator $\Kcal_t$ and its adjoint $\Kcal_t^*$ can be defined analogously via 
\begin{align}
 \Kcal_t \colon \R^N \xrightarrow{i} \C^N \xrightarrow{\Fcal} \C^N \xrightarrow{\Scal_t} \C^{M_t}
\end{align}
and 
\begin{align}
 \Kcal_t^* \colon \C^{M_t}  \xrightarrow{\Scal_t^*} \C^{N} \xrightarrow{\Fcal^{-1}} \C^N \xrightarrow{i^*} \R^N.
\end{align}
%
\begin{myrem}
 We would like to remark that we keep $N$ fixed for the dynamic scans, implying that the resolution of the $u_t$ stays the same over time. 
 We however theoretically keep the possibility that $N_0 > N$, i.e. that the prior has a higher resolution. 
 This would involve some downsampling techniques after the computation of the subgradient $p_0$.
 Furthermore, in general $M_t$ will be much smaller than $M_0$ such that the acquisition is sufficiently fast.
\end{myrem}
%
The reconstruction of the dynamic series then can be done via the following reconstruction scheme:
\begin{align}
   \arg \min_{\ubold} \sum_{t=1}^T \frac{\alpha_t}{2} \| \Kcal_t u_t - f_t\|_\C^2 + w_t \TV(u_t) + (1-w_t) \ICBTV^{p_0}(u_t, u_0 ) + \sum_{t=1}^{T-1} \frac{\gamma_t}{2} \| u_{t+1} - u_t \|^2,
   \label{eq:jr_fmri}
\end{align}
where $\ubold = [u_1, \dots, u_T]$ and $u_t \in \R^N$ for all $t = 1, \dots, T$.
We need to choose three sets of parameters: the regularization parameters $\alpha_t$ and $\gamma_t$, which control the tradeoff between data fidelity, temporal regularization, and $w_i \in [0,1]$, which controls the amount of total variation regularization versus similarity to the prior. 
They can be chosen equally for all time steps $t$, but can as well be adapted to when the activation is expected.

\subsection{Creating the operators}
The operators can be created the same way as above. 
The difference is that we now want to perform undersampling, i.e. we need to pick a different sampling scheme. 
A particularly interesting one is a sampling on radial spokes that are chosen according to the golden ratio [ref]. 
More precisely, we divide $\pi$ by the golden ratio $r = \frac{1 + \sqrt{5}}{2}$ to obtain 
\begin{align*}
 \frac{\pi}{r}  = \frac{2 * \pi}{1 + \sqrt{5}} \approx 1.9416.
\end{align*}
Then we choose the $k$-th spoke according to the angle 
\begin{align*}
 \varphi_k = \frac{k \pi}{r} \mod 2 \pi.
\end{align*}
This consecutively fills the entire $k$-space in an interleaving fashion.
For example, choosing 10 spokes per time step, we use spokes with the angles $\{\varphi_1, \dots, \varphi_{10} \}$ for time step $t=1$, and continue with angles $\{\varphi_{11}, \dots, \varphi_{20} \}$ for $t = 2$ and so on.
The same way as before, we can also create noisy data according to the respective sampling. 
We summarize the procedure in the following code: 
\lstinputlisting{matlab_includes/c_dyndata.m}
%
If $T$ is high, i.e. the number of time steps, for computational reasons it is smarter to divide the data set into {\it ascending and overlapping} bits $T_l$, i.e. $T_1 = \{1,\dots,10\}$, $T_2 = \{10,\dots,20\}$ and so on, such that we solve 
\begin{align}
   \arg \min_{\ubold} \sum_{t\in T_l} \frac{\alpha_t}{2} \| \Kcal_t u_t - f_t\|_\C^2 + w_t \TV(u_t) + (1-w_t) \ICBTV^{p_0}(u_t, u_0 ) + \sum_{t\in T_l \setminus \max\{T_l\}} \frac{\gamma_t}{2} \| u_{t+1} - u_t \|^2.
   \label{eq:jr_fmri_bits}
\end{align}
We can then choose the reconstruction parameters for every set (note that this is an arbitrary choice here, and that the regularization parameters can of course also vary within the sets).
\lstinputlisting{matlab_includes/c_parameters.m}

The last thing to do is to run the reconstruction: 
\lstinputlisting{matlab_includes/c_recon.m}
%
Here, we lowered the tolerance for `convergence' of the primal-dual residual. 
For a higher accuracy we can change this for the cost of a longer computation time.
If you have time, check that both energy and residual are very small. 
If energy and residual oscillate strongly, decrease the step sizes a little.

\section{Evaluation}
This section is dedicated to the evaluation of the results. 
I have some code available, which is however not perfect yet, and maybe not needed exactly now. 
I will add this as soon as we have figured out a nice way to visualize everything.

\section{Implementation details}
This section will to contain the details of the implementation, such as the primal-dual formulation and some further description of how the algorithms \mat{l2_tv.m} and \mat{jr_fmri.m} are realized and implemented.
This is still to do.
 




\end{document}
